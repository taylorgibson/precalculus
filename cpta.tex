%**************************************%
%* Generated from MathBook XML source *%
%*    on 2017-03-07T11:58:31-05:00    *%
%*                                    *%
%*   http://mathbook.pugetsound.edu   *%
%*                                    *%
%**************************************%
\documentclass[10pt,]{book}
%% Custom Preamble Entries, early (use latex.preamble.early)
%% Inline math delimiters, \(, \), need to be robust
%% 2016-01-31:  latexrelease.sty  supersedes  fixltx2e.sty
%% If  latexrelease.sty  exists, bugfix is in kernel
%% If not, bugfix is in  fixltx2e.sty
%% See:  https://tug.org/TUGboat/tb36-3/tb114ltnews22.pdf
%% and read "Fewer fragile commands" in distribution's  latexchanges.pdf
\IfFileExists{latexrelease.sty}{}{\usepackage{fixltx2e}}
%% Text height identically 9 inches, text width varies on point size
%% See Bringhurst 2.1.1 on measure for recommendations
%% 75 characters per line (count spaces, punctuation) is target
%% which is the upper limit of Bringhurst's recommendations
%% Load geometry package to allow page margin adjustments
\usepackage{geometry}
\geometry{letterpaper,total={340pt,9.0in}}
%% Custom Page Layout Adjustments (use latex.geometry)
%% This LaTeX file may be compiled with pdflatex, xelatex, or lualatex
%% The following provides engine-specific capabilities
%% Generally, xelatex and lualatex will do better languages other than US English
%% You can pick from the conditional if you will only ever use one engine
\usepackage{ifthen}
\usepackage{ifxetex,ifluatex}
\ifthenelse{\boolean{xetex} \or \boolean{luatex}}{%
%% begin: xelatex and lualatex-specific configuration
%% fontspec package will make Latin Modern (lmodern) the default font
\ifxetex\usepackage{xltxtra}\fi
\usepackage{fontspec}
%% realscripts is the only part of xltxtra relevant to lualatex 
\ifluatex\usepackage{realscripts}\fi
%% 
%% Extensive support for other languages
\usepackage{polyglossia}
\setdefaultlanguage{english}
%% Magyar (Hungarian)
\setotherlanguage{magyar}
%% Spanish
\setotherlanguage{spanish}
%% Vietnamese
\setotherlanguage{vietnamese}
%% end: xelatex and lualatex-specific configuration
}{%
%% begin: pdflatex-specific configuration
%% translate common Unicode to their LaTeX equivalents
%% Also, fontenc with T1 makes CM-Super the default font
%% (\input{ix-utf8enc.dfu} from the "inputenx" package is possible addition (broken?)
\usepackage[T1]{fontenc}
\usepackage[utf8]{inputenc}
%% end: pdflatex-specific configuration
}
%% Symbols, align environment, bracket-matrix
\usepackage{amsmath}
\usepackage{amssymb}
%% allow page breaks within display mathematics anywhere
%% level 4 is maximally permissive
%% this is exactly the opposite of AMSmath package philosophy
%% there are per-display, and per-equation options to control this
%% split, aligned, gathered, and alignedat are not affected
\allowdisplaybreaks[4]
%% allow more columns to a matrix
%% can make this even bigger by overriding with  latex.preamble.late  processing option
\setcounter{MaxMatrixCols}{30}
%%
%% Color support, xcolor package
%% Always loaded.  Used for:
%% mdframed boxes, add/delete text, author tools
\PassOptionsToPackage{usenames,dvipsnames,svgnames,table}{xcolor}
\usepackage{xcolor}
%%
%% Semantic Macros
%% To preserve meaning in a LaTeX file
%% Only defined here if required in this document
%% Subdivision Numbering, Chapters, Sections, Subsections, etc
%% Subdivision numbers may be turned off at some level ("depth")
%% A section *always* has depth 1, contrary to us counting from the document root
%% The latex default is 3.  If a larger number is present here, then
%% removing this command may make some cross-references ambiguous
%% The precursor variable $numbering-maxlevel is checked for consistency in the common XSL file
\setcounter{secnumdepth}{3}
%% Environments with amsthm package
%% Theorem-like environments in "plain" style, with or without proof
\usepackage{amsthm}
\theoremstyle{plain}
%% Numbering for Theorems, Conjectures, Examples, Figures, etc
%% Controlled by  numbering.theorems.level  processing parameter
%% Always need a theorem environment to set base numbering scheme
%% even if document has no theorems (but has other environments)
\newtheorem{theorem}{Theorem}[section]
%% Only variants actually used in document appear here
%% Style is like a theorem, and for statements without proofs
%% Numbering: all theorem-like numbered consecutively
%% i.e. Corollary 4.3 follows Theorem 4.2
%% Example-like environments, normal text
%% Numbering is in sync with theorems, etc
\theoremstyle{definition}
\newtheorem{example}[theorem]{Example}
%% Localize LaTeX supplied names (possibly none)
\renewcommand*{\chaptername}{Chapter}
%% Equation Numbering
%% Controlled by  numbering.equations.level  processing parameter
\numberwithin{equation}{section}
%% Raster graphics inclusion, wrapped figures in paragraphs
%% \resizebox sometimes used for images in side-by-side layout
\usepackage{graphicx}
%%
%% hyperref driver does not need to be specified
\usepackage{hyperref}
%% Hyperlinking active in PDFs, all links solid and blue
\hypersetup{colorlinks=true,linkcolor=blue,citecolor=blue,filecolor=blue,urlcolor=blue}
\hypersetup{pdftitle={Contemporary Pre-Calculus Through Applications}}
%% If you manually remove hyperref, leave in this next command
\providecommand\phantomsection{}
%% Graphics Preamble Entries
\usepackage{tikz}
\usetikzlibrary{backgrounds}
\usetikzlibrary{arrows,matrix}
\usetikzlibrary{snakes}
%% If tikz has been loaded, replace ampersand with \amp macro
%% extpfeil package for certain extensible arrows,
%% as also provided by MathJax extension of the same name
%% NB: this package loads mtools, which loads calc, which redefines
%%     \setlength, so it can be removed if it seems to be in the 
%%     way and your math does not use:
%%     
%%     \xtwoheadrightarrow, \xtwoheadleftarrow, \xmapsto, \xlongequal, \xtofrom
%%     
%%     we have had to be extra careful with variable thickness
%%     lines in tables, and so also load this package late
\usepackage{extpfeil}
%% Custom Preamble Entries, late (use latex.preamble.late)
%% Begin: Author-provided packages
%% (From  docinfo/latex-preamble/package  elements)
%% End: Author-provided packages
%% Begin: Author-provided macros
%% (From  docinfo/macros  element)
%% Plus three from MBX for XML characters

\newcommand{\lt}{<}
\newcommand{\gt}{>}
\newcommand{\amp}{&}
%% End: Author-provided macros
%% Title page information for book
\title{Contemporary Pre-Calculus Through Applications}
\author{Mathematics Department\\
North Carolina School of Science and Mathematics
}
\date{March 7, 2017}
\begin{document}
\frontmatter
%% begin: half-title
\thispagestyle{empty}
{\centering
\vspace*{0.28\textheight}
{\Huge Contemporary Pre-Calculus Through Applications}\\}
\clearpage
%% end:   half-title
%% begin: adcard
\thispagestyle{empty}
\null%
\clearpage
%% end:   adcard
%% begin: title page
%% Inspired by Peter Wilson's "titleDB" in "titlepages" CTAN package
\thispagestyle{empty}
{\centering
\vspace*{0.14\textheight}
%% Target for xref to top-level element is ToC
\addtocontents{toc}{\protect\hypertarget{cpta}{}}
{\Huge Contemporary Pre-Calculus Through Applications}\\[3\baselineskip]
{\Large Mathematics Department}\\[0.5\baselineskip]
{\Large North Carolina School of Science and Mathematics}\\[3\baselineskip]
{\Large March 7, 2017}\\}
\clearpage
%% end:   title page
%% begin: copyright-page
\thispagestyle{empty}
\vspace*{\stretch{2}}
\vspace*{\stretch{1}}
\null\clearpage
%% end:   copyright-page
%% begin: table of contents
%% Adjust Table of Contents
\setcounter{tocdepth}{1}
\renewcommand*\contentsname{Contents}
\tableofcontents
%% end:   table of contents
\mainmatter
\typeout{************************************************}
\typeout{Chapter 1 Exponential Functions}
\typeout{************************************************}
\chapter[{Exponential Functions}]{Exponential Functions}\label{chapter04}
\typeout{************************************************}
\typeout{Section 1.1 Recursive Functions}
\typeout{************************************************}
\section[{Recursive Functions}]{Recursive Functions}\label{chapter01-section01}
In a previous chapter we learned that a function is a special sets of ordered pairs.  In most of the examples in the preceeding chapters, functions were described by an algebraic expression that could be evaluated for a particular input value resulting in a unique output value. Such algebraic expressions are called closed form or explicit expressions.  For these functions, the relationship \(y=f(x)\) is used to show how the \(y\)-value is related to the given \(x\)-value. For example, the function \(f(x)=x^2+6x\) is an explicit function. This notation tells us that any particular numerical value for \(x\) is paired with the \(y\)-value equal to \(x^2+6x\). So 1 is paired with 7, since \(f(1)=(1)^2+6(1)=7\) , and \(-3\) is paired with \(-9\),since \(f(-3)=(-3)^2+6(-3)=-9\).%
\par
In this section we will investigate functions that are defined recursively. The domain values for these functions are positive whole numbers, and each range value is defined in terms of the preceding range value, rather than in terms of an \(x\)-value.%
\begin{example}[Ibuprofen in the blood stream]\label{example-1}
Joan has a headache and decides to take a 200mg ibuprofen tablet for pain relief.  The drug is absorbed into her system and stays in her system until the drug is metabolized and filtered out by the liver and kidneys.  Ibuprofen is rapidly metabolized.  Every four hours, Joan's body removes \(67\%\) of the ibuprofen that was in her body at the beginning of that four-hour time period.  How much of the ibuprofen will remain in her system \(24\) hours after taking the \(200\)mg tablet?%
\par\medskip\noindent%
\textbf{Solution.}\quad One way to generate values for the amount of ibuprofen in Joan's system is to use an iterative process.  In any iterative process the current value of a variable is used to determine the next value.  In this example, we generate a new amount of ibuprofen by subtracting the amount of ibuprofen filtered out of Joan's system from the amount that was previously there.  Since Joan begins with 200mg of ibuprofen, we write%
\begin{gather*}
D_0=0
\end{gather*}
%
\end{example}
\typeout{************************************************}
\typeout{Chapter 2 Logarithmic Functions}
\typeout{************************************************}
\chapter[{Logarithmic Functions}]{Logarithmic Functions}\label{chapter05}
\typeout{************************************************}
\typeout{Introduction  }
\typeout{************************************************}
In Jonathan Swift's Gulliver's Travels, the Lilliputians make new clothes for Gulliver.%
\typeout{************************************************}
\typeout{Section 2.1 }
\typeout{************************************************}
\section[{}]{}\label{chapter05-section01}
\end{document}